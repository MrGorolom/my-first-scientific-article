\documentclass[10pt,pdf,hyperref={unicode}]{beamer}

\mode<presentation>
{
\usetheme{boxes}
\beamertemplatenavigationsymbolsempty

\setbeamertemplate{footline}[page number]
\setbeamersize{text margin left=0.5em, text margin right=0.5em}
}

\usepackage[utf8]{inputenc}
\usepackage[english, russian]{babel}
\usepackage{bm}
\usepackage{multirow}
\usepackage{ragged2e}
\usepackage{indentfirst}
\usepackage{multicol}
\usepackage{subfig}
\usepackage{amsmath,amssymb}
\usepackage{enumerate}
\usepackage{mathtools}
\usepackage{comment}
\usepackage{multicol}
\usepackage{booktabs}
\usepackage{array}
\usepackage{makecell}

\usepackage[all]{xy}

\usepackage{tikz}
\usetikzlibrary{positioning,arrows}

\tikzstyle{name} = [parameters]
\definecolor{name}{rgb}{0.5,0.5,0.5}

\usepackage{caption}
\captionsetup{skip=0pt,belowskip=0pt}

\newtheorem{rustheorem}{Теорема}
\newtheorem{russtatement}{Утверждение}
\newtheorem{rusdefinition}{Определение}

% colors
\definecolor{darkgreen}{rgb}{0.0, 0.2, 0.13}
\definecolor{darkcyan}{rgb}{0.0, 0.55, 0.55}
\definecolor{darkred}{rgb}{0.6, 0.0, 0.0}
\definecolor{myblue}{rgb}{0.0, 0.2, 0.6}
\definecolor{mygray}{rgb}{0.9, 0.9, 0.9}

\AtBeginEnvironment{figure}{\setcounter{subfigure}{0}}

\captionsetup[subfloat]{labelformat=empty}

% Таблицы с границами
\newcolumntype{C}[1]{>{\centering\arraybackslash}m{#1}}
\newcolumntype{L}[1]{>{\raggedright\arraybackslash}m{#1}}
\newcolumntype{R}[1]{>{\raggedleft\arraybackslash}m{#1}}

%----------------------------------------------------------------------------------------------------------

\title[Скрининг деменции по речи]{Сравнение речевых тестов для ранней диагностики деменции}
\author{Д.С. Слободин, О.В. Сенько}

\institute[]{Московский государственный университет \\ Факультет ВМК}
\date[2024]{}

%---------------------------------------------------------------------------------------------------------
\begin{document}

\begin{frame}
\titlepage
\end{frame}

%----------------------------------------------------------------------------------------------------------
\begin{frame}{Клиническая проблема: необходимость ранней диагностики}
\begin{columns}[T]
\begin{column}{0.48\textwidth}
\begin{block}{Текущие ограничения}
\begin{itemize}
\item Поздняя диагностика (стадия деменции)
\item Субъективность клинических тестов
\item Высокая стоимость обследования
\item Недоступность в регионах
\end{itemize}
\end{block}
\end{column}

\begin{column}{0.48\textwidth}
\begin{block}{Наше решение}
\begin{itemize}
\item Объективный цифровой биомаркер
\item Быстрая запись (3-5 минут)
\item Автоматизированный анализ
\item Возможность удалённого скрининга
\end{itemize}
\end{block}
\end{column}
\end{columns}

\bigskip
\begin{center}
\color{darkred}
\textbf{Исследовательский вопрос:} Какой речевой тест эффективнее для скрининга?
\end{center}
\end{frame}

%----------------------------------------------------------------------------------------------------------
\begin{frame}{Уникальный датасет: 95 участников с клиническим подтверждением}
\begin{block}{Сложности сбора медицинских речевых данных}
\begin{itemize}
\item Клиническая верификация диагноза (неврологи + нейропсихологи)
\item Согласие пациентов с когнитивными нарушениями
\item Контроль условий записи
\item Баланс по полу и возрасту
\end{itemize}
\end{block}

\begin{table}[h]
\centering
\scriptsize
\begin{tabular}{|l|c|c|c|}
\hline
\textbf{Группа} & \textbf{n} & \textbf{Мужчины/Женщины} & \textbf{Возраст (лет)} \\
\hline
Здоровые & 32 & 14/18 & 68.2 ± 5.1 \\
MCI & 35 & 15/20 & 71.4 ± 6.3 \\
Деменция & 28 & 12/16 & 74.8 ± 7.2 \\
\hline
\multicolumn{4}{|c|}{\textbf{Ключевая особенность: каждый участник выполнил оба теста}} \\
\hline
\end{tabular}
\end{table}

\bigskip
\begin{center}
\color{darkgreen}
\textbf{Редкая возможность прямого сравнения на идентичной когорте}
\end{center}
\end{frame}

%----------------------------------------------------------------------------------------------------------
\begin{frame}{Распределение целевых переменных}
\begin{columns}[T]
\begin{column}{0.48\textwidth}
\begin{center}
\textbf{Регрессия: клинические шкалы}
\includegraphics[width=\textwidth]{target_distribution_regression.jpg}
\begin{itemize}
\item MMSE: 0-30
\item MoCA: 0-30  
\item CDR: 0-3
\end{itemize}
\end{center}
\end{column}

\begin{column}{0.48\textwidth}
\begin{center}
\textbf{Классификация: бинарные задачи}
\includegraphics[width=\textwidth]{target_distribution_classification.jpg}
\begin{itemize}
\item t\_MMSE: <26 vs ≥26
\item t\_MoCA: <26 vs ≥26
\item t\_CDR: <0.5 vs ≥0.5
\end{itemize}
\end{center}
\end{column}
\end{columns}

\bigskip
\begin{center}
\color{myblue}
\textbf{Две постановки задачи: регрессия и классификация}
\end{center}
\end{frame}

%----------------------------------------------------------------------------------------------------------
\begin{frame}{Экспериментальная установка}
\begin{table}[h]
\centering
\scriptsize
\begin{tabular}{|l|c|c|c|}
\hline
\textbf{Компонент} & \textbf{Опция 1} & \textbf{Опция 2} & \textbf{Опция 3} \\
\hline
\textbf{Речевой тест} & Чтение (R) & Описание (D) & - \\
\hline
\textbf{Признаки (openSMILE)} & eGeMAPS (88) & GeMAPS (62) & EMOBASE (988) \\
\hline
\textbf{Модель (ML)} & Random Forest & Gradient Boosting & Log. Regression \\
\hline
\textbf{Валидация} & \multicolumn{3}{c|}{Leave-One-Out (95 фолдов)} \\
\hline
\textbf{Целевые шкалы} & \multicolumn{3}{c|}{MMSE, MoCA, CDR} \\
\hline
\end{tabular}
\end{table}

\begin{block}{Гипотеза и альтернатива}
\begin{itemize}
\item \textbf{Гипотеза:} Меньше вариативность → лучше диагностика (чтение лучше)
\item \textbf{Альтернатива:} Больше когнитивной нагрузки → больше сигнала (описание лучше)
\end{itemize}
\end{block}
\end{frame}

%----------------------------------------------------------------------------------------------------------
\begin{frame}{Результаты регрессии: сравнение модальностей}
\begin{table}[h]
\centering
\scriptsize
\begin{tabular}{|l|l|c|c|c|}
\hline
\textbf{Модальность} & \textbf{Шкала} & \textbf{RMSE} & \textbf{R²} & \textbf{Лучшая модель} \\
\hline
\multirow{3}{*}{\textbf{Чтение (R)}} & MMSE & 3.36 & 0.15 & CatBoost \\
 & MoCA & 4.93 & -0.04 & CatBoost \\
 & CDR & 0.42 & 0.04 & CatBoost \\
\hline
\multirow{3}{*}{\textbf{Описание (D)}} & MMSE & 3.50 & 0.07 & CatBoost \\
 & MoCA & 4.50 & \cellcolor{mygray}0.14 & Random Forest \\
 & CDR & 0.40 & \cellcolor{mygray}0.15 & Random Forest \\
\hline
\end{tabular}
\end{table}

\begin{block}{Ключевой вывод}
\begin{itemize}
\item Спонтанное описание превосходит чтение для MoCA и CDR
\item Чтение немного лучше для MMSE
\item MoCA и CDR оценивают исполнительные функции — лучше выявляются в спонтанной речи
\end{itemize}
\end{block}
\end{frame}

%----------------------------------------------------------------------------------------------------------
\begin{frame}{Результаты классификации: ROC-AUC}
\begin{table}[h]
\centering
\scriptsize
\begin{tabular}{|l|l|c|c|c|c|}
\hline
\textbf{Модальность} & \textbf{Задача} & \textbf{ROC-AUC} & \textbf{F1-score} & \textbf{Precision} & \textbf{Лучшая модель} \\
\hline
\multirow{3}{*}{\textbf{Чтение (R)}} & t\_MMSE & 0.79 & 0.46 & 0.42 & Log. Regression \\
 & t\_MoCA & 0.39 & 0.80 & 0.73 & CatBoost \\
 & t\_CDR & 0.77 & 0.49 & 0.46 & Log. Regression \\
\hline
\multirow{3}{*}{\textbf{Описание (D)}} & t\_MMSE & 0.77 & 0.36 & 0.35 & Log. Regression \\
 & t\_MoCA & \cellcolor{mygray}0.60 & \cellcolor{mygray}0.84 & 0.74 & Random Forest \\
 & t\_CDR & 0.74 & 0.38 & 0.67 & Random Forest \\
\hline
\end{tabular}
\end{table}

\begin{block}{Основные наблюдения}
\begin{itemize}
\item \textbf{t\_MoCA:} Описание значительно лучше (0.60 vs 0.39)
\item \textbf{t\_MMSE:} Обе модальности показывают высокий AUC (>0.77)
\item \textbf{Лучшая модель:} Random Forest для сложных задач
\item \textbf{Практическая значимость:} Описание лучше для раннего выявления MCI
\end{itemize}
\end{block}
\end{frame}

%----------------------------------------------------------------------------------------------------------
\begin{frame}{Оптимальная техническая конфигурация}
\begin{table}[h]
\centering
\scriptsize
\begin{tabular}{|l|c|c|c|c|}
\hline
\textbf{Набор признаков} & \textbf{Лучшая модель} & \textbf{ROC-AUC} & \textbf{F1-score} & \textbf{Число признаков} \\
\hline
EGEMAPS\_D & Random Forest & 0.7428 & 0.3429 & 88 \\
\hline
EGEMAPS\_R & Gradient Boosting & 0.7421 & 0.4500 & 88 \\
\hline
EMOBASE\_R & Random Forest & \cellcolor{mygray}0.8062 & 0.3636 & 988 \\
\hline
GEMAPS\_D & Random Forest & 0.7246 & 0.4211 & 62 \\
\hline
GEMAPS\_R & Log. Regression & 0.7905 & 0.4103 & 62 \\
\hline
COMPARE\_R & Random Forest & 0.7923 & 0.4000 & 6371 \\
\hline
\end{tabular}
\end{table}

\begin{block}{Технические выводы}
\begin{itemize}
\item \textbf{Random Forest} побеждает в 4 из 6 конфигураций
\item \textbf{eGeMAPS (88 признаков)} конкурирует с 6371 признаками
\item \textbf{Лучший результат:} EMOBASE\_R + Random Forest (AUC 0.8062)
\item \textbf{Для клиники:} eGeMAPS достаточно, важна интерпретируемость
\end{itemize}
\end{block}
\end{frame}

%----------------------------------------------------------------------------------------------------------
\begin{frame}{Клинические рекомендации и выводы}
\begin{columns}[T]
\begin{column}{0.48\textwidth}
\begin{block}{Рекомендации по выбору теста}
\begin{itemize}
\item \textbf{MoCA/CDR оценка:} спонтанное описание
\item \textbf{MMSE скрининг:} стандартизированное чтение
\item \textbf{Полный скрининг:} двухуровневый протокол
\end{itemize}
\end{block}
\end{column}

\begin{column}{0.48\textwidth}
\begin{block}{Технические рекомендации}
\begin{itemize}
\item \textbf{Признаки:} eGeMAPS (88 параметров)
\item \textbf{Модель:} Random Forest
\item \textbf{Валидация:} LOO + статистические тесты
\end{itemize}
\end{block}
\end{column}
\end{columns}

\bigskip
\begin{block}{Основные научные выводы}
\begin{enumerate}
\item Доказана эффективность речевого анализа (AUC до 0.81)
\item Опровергнута гипотеза о преимуществе стандартизации
\item Установлена специфичность речевых задач
\item Предложен практический клинический протокол
\end{enumerate}
\end{block}

\bigskip
\begin{center}
\color{darkcyan}
\textbf{Речь — перспективный цифровой биомаркер для массового скрининга}
\end{center}
\end{frame}

%----------------------------------------------------------------------------------------------------------
\begin{frame}{Вопросы?}
\begin{center}
\Large
Спасибо за внимание!

\vspace{0.5cm}
\begin{small}
Уникальный датасет с клинической верификацией \\
Прямое сравнение речевых модальностей \\
Практические рекомендации для клиники
\end{small}

\vspace{1cm}
\begin{tiny}
Код и данные: \url{https://github.com/MrGorolom/my-first-scientific-article}
\end{tiny}
\end{center}
\end{frame}

%----------------------------------------------------------------------------------------------------------

\end{document}